% Options for packages loaded elsewhere
\PassOptionsToPackage{unicode}{hyperref}
\PassOptionsToPackage{hyphens}{url}
\PassOptionsToPackage{dvipsnames,svgnames,x11names}{xcolor}
%
\documentclass[
]{article}
\usepackage{amsmath,amssymb}
\usepackage{lmodern}
\usepackage{iftex}
\ifPDFTeX
  \usepackage[T1]{fontenc}
  \usepackage[utf8]{inputenc}
  \usepackage{textcomp} % provide euro and other symbols
\else % if luatex or xetex
  \usepackage{unicode-math}
  \defaultfontfeatures{Scale=MatchLowercase}
  \defaultfontfeatures[\rmfamily]{Ligatures=TeX,Scale=1}
\fi
% Use upquote if available, for straight quotes in verbatim environments
\IfFileExists{upquote.sty}{\usepackage{upquote}}{}
\IfFileExists{microtype.sty}{% use microtype if available
  \usepackage[]{microtype}
  \UseMicrotypeSet[protrusion]{basicmath} % disable protrusion for tt fonts
}{}
\makeatletter
\@ifundefined{KOMAClassName}{% if non-KOMA class
  \IfFileExists{parskip.sty}{%
    \usepackage{parskip}
  }{% else
    \setlength{\parindent}{0pt}
    \setlength{\parskip}{6pt plus 2pt minus 1pt}}
}{% if KOMA class
  \KOMAoptions{parskip=half}}
\makeatother
\usepackage{xcolor}
\usepackage[margin=1in]{geometry}
\usepackage{graphicx}
\makeatletter
\def\maxwidth{\ifdim\Gin@nat@width>\linewidth\linewidth\else\Gin@nat@width\fi}
\def\maxheight{\ifdim\Gin@nat@height>\textheight\textheight\else\Gin@nat@height\fi}
\makeatother
% Scale images if necessary, so that they will not overflow the page
% margins by default, and it is still possible to overwrite the defaults
% using explicit options in \includegraphics[width, height, ...]{}
\setkeys{Gin}{width=\maxwidth,height=\maxheight,keepaspectratio}
% Set default figure placement to htbp
\makeatletter
\def\fps@figure{htbp}
\makeatother
\setlength{\emergencystretch}{3em} % prevent overfull lines
\providecommand{\tightlist}{%
  \setlength{\itemsep}{0pt}\setlength{\parskip}{0pt}}
\setcounter{secnumdepth}{-\maxdimen} % remove section numbering
\usepackage{booktabs}
\usepackage{longtable}
\usepackage{array}
\usepackage{multirow}
\usepackage{wrapfig}
\usepackage{float}
\usepackage{colortbl}
\usepackage{pdflscape}
\usepackage{tabu}
\usepackage{threeparttable}
\usepackage{threeparttablex}
\usepackage[normalem]{ulem}
\usepackage{makecell}
\usepackage{xcolor}
\ifLuaTeX
  \usepackage{selnolig}  % disable illegal ligatures
\fi
\IfFileExists{bookmark.sty}{\usepackage{bookmark}}{\usepackage{hyperref}}
\IfFileExists{xurl.sty}{\usepackage{xurl}}{} % add URL line breaks if available
\urlstyle{same} % disable monospaced font for URLs
\hypersetup{
  pdftitle={Analysis of Spotify Attributes for Audio Tracks},
  pdfauthor={Sigal Falk, ID 206682031 \& Carmel Baris, ID 318455276},
  colorlinks=true,
  linkcolor={Maroon},
  filecolor={Maroon},
  citecolor={Blue},
  urlcolor={blue},
  pdfcreator={LaTeX via pandoc}}

\title{Analysis of \emph{Spotify} Attributes for Audio Tracks}
\author{Sigal Falk, ID 206682031 \& Carmel Baris, ID 318455276}
\date{2022-12-01}

\begin{document}
\maketitle

\begin{center}\rule{0.5\linewidth}{0.5pt}\end{center}

\hypertarget{chosen-dataset}{%
\subsubsection{Chosen Dataset}\label{chosen-dataset}}

The Swedish company, \emph{Spotify}, is an audio streaming and media
services provider. Its digital services include a public REST-API which
reveals the company's records of artists, albums, playlists, and songs.
Specifically, the
\href{https://developer.spotify.com/documentation/web-api/reference/\#/operations/get-several-audio-features}{\emph{``Track's
Audio Features''}} endpoint provides quantitative attributes for each
requested song. We selected a dataset based on this endpoint, which
includes over 170K songs dated from 1921 till
2020\href{https://data.world/babarory/spotify-dataset-1921-2020}{babarory/spotify-dataset-1921-2020}.
In our analysis we focused solely on the years 2019-2020 and on two
attributes:

A. \textbf{Valence} index ranging from 0.0 to 0.1 and which describes
the musical positiveness conveyed by a track. An audio track ranked high
in terms of valence is considered to convey more positive emotions
(e.g.~happy, cheerful, euphoric), whereas tracks with low valence are
considered to convey more negative emotions (e.g.~sad, depressed,
angry).

B. \textbf{Danceability} index, which describes the suitability of the
song track for dancing and is based on the combination of several
musical elements including rhythm, rhythm stability, the intensity of
the music beats, and the general regularity of the audio track. The
index ranges from 0.0 which describes an audio track that is unsuitable
for dancing, to 1.0 which refers to a track that can be danced to with
great ease.

We should note that both attributes are internal to \emph{Spotify}'s
databases and services, meaning they are not based on a known standard
and therefore don't use publicly recognized measuring units. In this
report we'll refer to these attributes as Valence Index (abbr. VI) and
Danceability Index (DI).

We now proceed to analyse the data at hand.

\hypertarget{measures-of-central-location-and-dispersion}{%
\subsubsection{Measures of Central Location and
Dispersion}\label{measures-of-central-location-and-dispersion}}

\hypertarget{danceability-index}{%
\subparagraph{Danceability Index}\label{danceability-index}}

\begin{verbatim}
##    Min. 1st Qu.  Median    Mean 3rd Qu.    Max. 
##   0.000   0.415   0.548   0.537   0.668   0.988
\end{verbatim}

\hypertarget{valence-index}{%
\subparagraph{Valence Index}\label{valence-index}}

\begin{verbatim}
##    Min. 1st Qu.  Median    Mean 3rd Qu.    Max. 
##   0.000   0.317   0.540   0.529   0.747   1.000
\end{verbatim}

Above, the minimal value for both DI and VI is 0.0, giving us reason to
believe our dataset may include records with missing data. A quick
survey of the CSV file itself confirms our suspicion. To refrain from
diverting our attention, we have decided to exclude these records from
our current report by filtering out all songs that have a DI of zero.

Having removed 166,684 records, we recalculate the descriptive
statistics for DI and VI:

\hypertarget{danceability-index-1}{%
\subparagraph{Danceability Index}\label{danceability-index-1}}

\begin{verbatim}
##    Min. 1st Qu.  Median    Mean 3rd Qu.    Max. 
##   0.067   0.574   0.693   0.671   0.788   0.980
\end{verbatim}

\hypertarget{valence-index-1}{%
\subparagraph{Valence Index}\label{valence-index-1}}

\begin{verbatim}
##    Min. 1st Qu.  Median    Mean 3rd Qu.    Max. 
## 0.00001 0.31300 0.48500 0.48157 0.65000 0.97900
\end{verbatim}

\hypertarget{charts}{%
\subsubsection{Charts}\label{charts}}

In this section, we turn to some charts that will further help us
understand our data.

First, we use a histogram to review the distribution of each audio
attribute.

\includegraphics[width=0.5\linewidth]{spotify_danceability_valence_files/figure-latex/plots-1}
\includegraphics[width=0.5\linewidth]{spotify_danceability_valence_files/figure-latex/plots-2}

We see that the VI has a symmetric normal distribution, evident also
from the brief statistics from earlier, in which the median and mean
were nearly identical (0.485 versus 0.481, respectively). The DI, on the
other hand, has a slightly negatively skewed (left-skewed) distribution,
i.e.~its extreme values are concentrated on the lower end of the index.
This aligns with the relatively high values of its first quantile (0.5
DI) and of its third quantile (0.8 DI), which indicate that most of the
songs in our dataset are significantly suitable for dancing.

The standard deviations in the Valence Index and in the Danceability
Index are \(SD_{valence}=\) 0.227 and \(SD_{dance}=\) 0.163,
respectively.

\pagebreak

\hypertarget{linear-regression-model}{%
\subsubsection{Linear Regression Model}\label{linear-regression-model}}

We are interested in investigating the relationship between the VI as a
predictor variable, and the DI as a response variable. According to our
hypothesis, if a song conveys relatively positive emotions (higher
valence) it will also be easier to dance to; in opposition, it will be
harder to dance to a song conveying negative emotions. To test our
hypothesis, we'll create a scatter plot describing how much the
Danceability Index changes with respect to changes in the Valence Index.

\includegraphics{spotify_danceability_valence_files/figure-latex/regression-1.pdf}

To the plot we added indicators of the average attribute values (gray
for VI, blue for DI), as well as a least-squares linear regression line
(red) which marks the observations that are closest to the average.
Let's take a closer look at the coefficients of the regression line:

\(\hat a =\) 0.29 is the line's slope and describes the growth in
Danceability Index units, if the song's Valence increases by one VI
unit.

\(\hat b =\) 0.531 is the line's intercept and describes the size of a
song's Danceability Index, when the song's Valence Index isn't taken
into consideration (as it reaches a negligible value). In other words, a
song's basic DI score is 0.29, and it increases with every additional VI
unit.

\hypertarget{evaluating-goodness-of-prediction}{%
\subsubsection{Evaluating Goodness of
Prediction}\label{evaluating-goodness-of-prediction}}

Based on our sampled observations, we want to get an idea of the
precision rate of our simplified linear regression model
(\(\hat y = \hat a x + \hat b\)) using the following estimators.

\hypertarget{root-mean-square-error-rmse}{%
\subparagraph{Root-Mean-Square Error
(RMSE)}\label{root-mean-square-error-rmse}}

\(\text {RMSE}=\) 0.1495 DI units. Compared to the general standard
deviations of our attributes, the RMSE estimation is relatively low,
indicating that most observations in our sample deviate only slightly
from the average Danceability Index score (the regression line). This
suggests that our model exhibits a consistent relationship between the
VI predictors and their respective DI observations.

\hypertarget{pearson-correlation-coefficient}{%
\subparagraph{Pearson Correlation
Coefficient}\label{pearson-correlation-coefficient}}

\(\ {\displaystyle \rho \ _{_{_{_{{\text VI},{\text DI}}}}}={\frac{\operatorname {cov} (VI,D I)}{\sigma _{_{_{_{{\text VI}}}}} \sigma _{_{_{_{{\text DI}}}}}}}}=\)
0.403, somewhat weak yet \textbf{positive} correlation: the higher a
song's VI, its expected DI score is higher.

\hypertarget{coefficient-of-determination-r-squared-indicator}{%
\subparagraph{Coefficient of Determination (R-Squared
Indicator)}\label{coefficient-of-determination-r-squared-indicator}}

\(R^2:= {\rho}^2 (VI,DI)=\) 0.163 in terms of ``correlation strength'',
so to speak. Here too, the somewhat low value indicates a relatively
weak, albeit positive, correlation. In each new sample of songs, the
variance in Valence Index values will somewhat affect the variance in
Danceability Index values.

To conclude, in a given sample of songs, our Valence-based model can
account for almost 0.163 of the sample's variance in Danceability. It
seems that as far as \emph{Spotify} is concerned, a song's Valence Index
is a considerable (albeit not exclusive) factor in the song's
Danceablity score. To demonstrate, if a song has a score of 0.9 on the
Valence Index, we expect it will reach an estimate of 0.793 on the
Danceability Index scale.

\hypertarget{outlier-observations}{%
\subsubsection{Outlier Observations}\label{outlier-observations}}

Below is a brief analysis of the observations that were distanced the
most from the sample mean.

\begin{table}
\centering\begingroup\fontsize{10}{12}\selectfont

\begin{tabular}[t]{l|r|r|r|r}
\hline
name & valence & danceability & predictions & residuals\\
\hline
Boötes & 0.04 & 0.07 & 0.54 & -0.48\\
\hline
Andromeda & 0.40 & 0.17 & 0.65 & -0.48\\
\hline
Organ Prelude, No. 1 "The Phoenix" & 0.04 & 0.07 & 0.54 & -0.47\\
\hline
Sleep Music With White Noise & 0.03 & 0.07 & 0.54 & -0.47\\
\hline
\end{tabular}
\endgroup{}
\end{table}

\pagebreak

\end{document}
